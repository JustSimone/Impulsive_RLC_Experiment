\documentclass{article}[a4paper]
\usepackage{graphicx} % Required for inserting images
\usepackage{amsmath, amssymb}
\usepackage{mathtools}
\usepackage{derivative}

\usepackage{amsfonts}

\newcommand*{\eqdef}{\ensuremath{\overset{\mathclap{\text{\tiny def}}}{=}}}

\title{Exercise sheet Geometry in Physics - 1}
\author{Simone Coli}
\date{October $26^{\mbox{\small th}}$, 2023} 

\begin{document}
\maketitle
\section{Solution} %Exercise 1
    \subsection*{a)}
        Let $\mathbf{x, h} \in V$ then by definition of differentiability for $f(\mathbf{x}) = A(\mathbf{h})$
        \[
            A(\mathbf{x + h}) = A(\mathbf{x}) + A'(\mathbf{h}) + o(|\mathbf{h}|),
        \]
        such that $||\mathbf{h}||$ is small and $A' \in L(V, W)$. Since $A \in L(V, W)$, then it is linear and,
        \[
            A(\mathbf{x}) + A(\mathbf{h}) = A(\mathbf{x}) + A'(\mathbf{h}) + o(|\mathbf{h}|).
        \]
        From which $Df_{|W}(\mathbf{h}) = A(\mathbf{h})$. In the same way we can calculate the second derivative:
        \[
            Df_{|W}(\mathbf{x+h}) = A(\mathbf{x + h}) = A(\mathbf{x}) + A''(\mathbf{h}) + o(|\mathbf{h}|),
        \]
        meaning that $D^2f_{|W}(\mathbf{h}) = A(\mathbf{h})$ amd $D^3f_{|W}(\mathbf{h}) = A(\mathbf{h})$.

        For a bilinear map $g(\mathbf{x}) = b(\mathbf{x,x})$, with $b \in L(V,V;W)$ the definition of differentiability implies:
        \[
            b(\mathbf{x+h,x+h'}) = b(\mathbf{x,x}) + b'(\mathbf{h,h'}) + o(|{|\mathbf{h}|}^2+{|\mathbf{h'}|}^2|),
        \]
        where $\mathbf{h} \in \mathbb{V}$. Because of the bilinearity
        \[
            \begin{split}
                b(\mathbf{x+h,x+h}) &= b(\mathbf{x,x}) + b(\mathbf{h,x}) + b(\mathbf{x,h}) + b(\mathbf{h,h}) \\
                &= b(\mathbf{x,x}) + b'(\mathbf{h,h}) + o(|\mathbf{h}|),
            \end{split}
        \]
        meaning that $Dg_{|W}(\mathbf{h,h}) = 2b(\mathbf{h,x})$.
        Iterating this process we still get
        \[
            D^2 = 4b(x, h)
        \]
        \[
            D^3 = 8b(x, h)
        \]
    \subsection*{b)}
    Since $\mathbf{x} \in V= \mathbb{R}^2$ then it is a two-dimensional vector, meaning that $f \in V^*$ so that:
    \[
        f(\mathbf{x}) = 
        \begin{pmatrix}
             c & d
        \end{pmatrix}
        \begin{pmatrix}
            a\\
            b
        \end{pmatrix}
    \]
    where $\mathbf{x}=
    \begin{pmatrix}
        a\\
        b
    \end{pmatrix}$
   and with $a,b,c,d \in \mathbb{R}$.

   Since $g(\mathbf{x}) = b(\mathbf{x,x})$ this has to be of the form 
   \[
    g(\mathbf{x}) = b(\mathbf{x,x})=
    \begin{pmatrix}
        a & b
    \end{pmatrix}
    \begin{pmatrix}
        c & d\\
        e & f\\
    \end{pmatrix}
    \begin{pmatrix}
        a\\
        b
    \end{pmatrix}
   \]
   where $a,b,c,d,e,f \in \mathbb{R}$.
\section{Solution} %Exercise 2
    Let us consider a map $f: M_{n\times n} (\mathbb{R}) \to \mathbb{R}$ such that $A \to \det{A}$, then by definition of differentiability, we have that:
    \[
        f(A + H) = \det(A+H) = f(A) + f'(H) + o(||H||).
    \]
    The definition of the determinant for an $n\times n$ matrix is:
    \[
        T = \sum_{ \in S_n} \mbox{sg}(\sigma)\, a_{1, \sigma(1)} \cdots a_{n, \sigma(n)},
    \]
    If we calculate the derivative of our map in $A=I$, where $I$ is the identity, we get that:
    \[
        \begin{split}
            Df_{|I}(H) &= 
            \begin{vmatrix}
                h+1 & h & \cdots & h\\
                h & h+1 & \cdots & h\\
                \vdots & \vdots & \ddots & \vdots\\
                h & h & \cdots & h+1
            \end{vmatrix} - 1 =\\
            &= 1 + h + h + \cdots + h + h^2 + \cdots -1
        \end{split}
    \]
    There are just n terms of the sum that have a linear dependence from $h$ that is $Df_{|I}(H) = nh$ which is the trace of the $H$ matrix. therefore
    \[
        Df_{|I}(H) = \mbox{tr}(H) 
    \]
\section{Solution} %Exercise 3
Let us consider an implicit representation of the curve $\gamma(t)$, that is a function $h(x(t), y(t)) = 0$ where $t \in I$. If we assume this function to regular then
\[
    Dh_{|t_0} (t) = \left. \pdv{h}{x} \pdv{x}{t} \right|_{t_0} + \left. \pdv{h}{y} \pdv{y}{t} \right|_{t_0} \neq 0\; ,
\]
since by hypothesis $x'(t_0) \neq 0$ and $y'(t_0) \neq 0$. By the implicit function theorem:
\[
    \exists\, J \subset I\; :\; t_0 \in J 
\]
and such that either
\[
    \mbox{{\small graph}}(g_1) = \{ (x(t), g_1(x(t))),\; t \in J \} \eqdef \gamma_{|J}(t)\, ,
\]
\[
    \mbox{{\small graph}}(g_2) = \{ (g_2(y(t)), y(t)),\; t \in J \} \eqdef \gamma_{|J}(t)\, .
\]
So that $y(t) = g_1(x(t))$ and $x(t) = g_2 (y(t))$. The two functions are therefore one the inverse of the other.
\section{Solution} %Exercise 4
    \subsection*{a)}
        To prove the orthogonality of the two vectors we just need to show that their (standard) scalar product is zero. Let $\partial_\theta$ and $\partial_t$ be:
        \[
            \partial_\theta f =
            \begin{pmatrix}
                -r(t) \sin(\theta)\\
                r(t) \cos(\theta)\\
                0
            \end{pmatrix}, \;
            \partial_t = \begin{pmatrix}
                r'(t) \cos(\theta)\\
                r'(t) \sin(\theta)\\
                h'(t)
            \end{pmatrix},
        \]
        then:
        \[
            \langle \partial_\theta , \partial_t \rangle = - r(t)r'(t) \sin(\theta) \cos(\theta) + r(t)r'(t) \sin(\theta) \cos(\theta) + 0 = 0
        \]
    \subsection*{b)}
        If we assume $r(t) = 1/\mbox{cosh}(t)$ and $h(t) = tanh(t)$, then we have that:
        \[
            \left\| \pdv{f}{\theta} \right\| = \sqrt{(1+t^2)(\cos^2{\theta} + \sin^2{\theta})} = \sqrt{(1+t^2)}
        \]
        \[
            \left\| \pdv{f}{t} \right\| = \sqrt{\frac{t^2}{1-t^2} + 1} = \sqrt{\frac{t^2+1-t^2}{1-t^2}} = \frac{1}{\sqrt{1+t^2}}
        \]
        Which proves that:
        \[
            \left\| \pdv{f}{\theta} \right\|  \left\| \pdv{f}{t} \right\| = 1
        \]
    \subsection*{c)}
        If we assume $r(t) = 1/\mbox{cosh}(t)$ and $h(t) = \mbox{tanh}(t)$, then we have that:
            \[
                \left\| \pdv{f}{\theta} \right\| = \frac{1}{\mbox{cosh}^2(t)} (\sin^2{\theta} + \cos^2{\theta}) = \frac{1}{\mbox{cosh}^2(t)}
            \]
            \[\
                \begin{split}
                    \left\| \pdv{f}{t} \right\| &=  \frac{\mbox{sinh}^2(t)}{\mbox{cosh}^4(t)}\; (\sin^2{\theta} + \cos^2{\theta}) + \; \frac{1}{\mbox{cosh}^4(t)} =\\
                    &= \frac{\mbox{sinh}^2(t) + \mbox{cosh}^2(t) - \mbox{sinh}^2(t)}{\mbox{cosh}^4(t)} =\\
                    &=\frac{\mbox{cosh}^2(t)}{\mbox{cosh}^4(t)} =\frac{1}{\mbox{cosh}^2(t)}
                \end{split}
            \]
            which proves that, given these definitions of $r(t)$ and $h(t)$:
            \[
                \left\| \pdv{f}{\theta} \right\| = \left\| \pdv{f}{t} \right\|
            \]
\end{document}