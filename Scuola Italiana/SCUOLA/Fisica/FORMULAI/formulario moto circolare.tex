\documentclass[a4paper]{article}
\usepackage[T1]{fontenc}
\usepackage[utf8]{inputenc}
\usepackage[italian]{babel}
\usepackage{amsmath,tikz}
\usetikzlibrary{calc}
\newenvironment{sistema}%
{\left\lbrace\begin{array}{@{}l@{}}}%
{\end{array}\right.}
\newcommand{\tikzmark}[1]{\tikz[overlay,remember picture] \node (#1) {};}
\begin{document}
\begin{center}
{\Huge Formulario Moto Armonico}
\vspace{0,75cm}
\end{center}
{\huge -Moto circolare uniforme}
\\
\\
\begin{Large}
\hspace* {1,5cm} $\omega = \frac{\Delta \alpha}{\Delta t} $ (rad/s)
\hspace {1,5cm} $\omega = \frac{2*\pi}{T}$
\\
\\
\hspace*{1,5cm} $a_{c} = \frac{v^{2}}{r} = \omega^{2}*r$
\hspace{1,5cm} $F_{c} = m*\frac{v^{2}}{r} = m*\omega^{2}*r $
\end{Large}
{\huge - Moto armonico}
\\
\\
\begin{Large}
\hspace* {0,75cm} \textit{ampiezza = d;}\\
\hspace*{0,75cm} \textit{periodo = T};\\
\hspace*{0,75cm}\textit{frequenza = f =$ \frac{1}{T} $};
\\
\\
\hspace*{1,5cm}$s = r * cos(\omega*t)$
\\
\\
\hspace*{1,5cm}$v = -v_{0} * sen(\omega * t)$ dove $v_{0} = (\omega * r)$
\\
\\
\hspace*{1,5cm}$\overrightarrow{a} = -\omega^{2}*\overrightarrow{s} $
\\
\\
{\huge - Moto armonico di una molla}
\\
\\
\hspace*{1,5cm}$a = -\frac{k}{m}*r = -\omega^{2}*s$
\\
\\
\hspace*{1,5cm}$\omega^{2} = \frac{k}{m} \rightarrow \frac{2*\pi}{T} = \sqrt{\frac{k}{m}} \rightarrow T= 2*\pi*\sqrt{\frac{m}{k}}$
\\
\\
{\huge -Moto del pendolo}
\\
\\
\hspace*{1,5cm}$\omega^{2} = \frac{g}{l} \rightarrow \frac{2*\pi}{T} = \sqrt{\frac{g}{l}} \rightarrow T= 2*\pi*\sqrt{\frac{l}{g}}$
\end{Large}
\end{document}